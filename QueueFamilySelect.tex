\documentclass{article}
%% ODER: format ==         = "\mathrel{==}"
%% ODER: format /=         = "\neq "
%
%
\makeatletter
\@ifundefined{lhs2tex.lhs2tex.sty.read}%
  {\@namedef{lhs2tex.lhs2tex.sty.read}{}%
   \newcommand\SkipToFmtEnd{}%
   \newcommand\EndFmtInput{}%
   \long\def\SkipToFmtEnd#1\EndFmtInput{}%
  }\SkipToFmtEnd

\newcommand\ReadOnlyOnce[1]{\@ifundefined{#1}{\@namedef{#1}{}}\SkipToFmtEnd}
\usepackage{amstext}
\usepackage{amssymb}
\usepackage{stmaryrd}
\DeclareFontFamily{OT1}{cmtex}{}
\DeclareFontShape{OT1}{cmtex}{m}{n}
  {<5><6><7><8>cmtex8
   <9>cmtex9
   <10><10.95><12><14.4><17.28><20.74><24.88>cmtex10}{}
\DeclareFontShape{OT1}{cmtex}{m}{it}
  {<-> ssub * cmtt/m/it}{}
\newcommand{\texfamily}{\fontfamily{cmtex}\selectfont}
\DeclareFontShape{OT1}{cmtt}{bx}{n}
  {<5><6><7><8>cmtt8
   <9>cmbtt9
   <10><10.95><12><14.4><17.28><20.74><24.88>cmbtt10}{}
\DeclareFontShape{OT1}{cmtex}{bx}{n}
  {<-> ssub * cmtt/bx/n}{}
\newcommand{\tex}[1]{\text{\texfamily#1}}	% NEU

\newcommand{\Sp}{\hskip.33334em\relax}


\newcommand{\Conid}[1]{\mathit{#1}}
\newcommand{\Varid}[1]{\mathit{#1}}
\newcommand{\anonymous}{\kern0.06em \vbox{\hrule\@width.5em}}
\newcommand{\plus}{\mathbin{+\!\!\!+}}
\newcommand{\bind}{\mathbin{>\!\!\!>\mkern-6.7mu=}}
\newcommand{\rbind}{\mathbin{=\mkern-6.7mu<\!\!\!<}}% suggested by Neil Mitchell
\newcommand{\sequ}{\mathbin{>\!\!\!>}}
\renewcommand{\leq}{\leqslant}
\renewcommand{\geq}{\geqslant}
\usepackage{polytable}

%mathindent has to be defined
\@ifundefined{mathindent}%
  {\newdimen\mathindent\mathindent\leftmargini}%
  {}%

\def\resethooks{%
  \global\let\SaveRestoreHook\empty
  \global\let\ColumnHook\empty}
\newcommand*{\savecolumns}[1][default]%
  {\g@addto@macro\SaveRestoreHook{\savecolumns[#1]}}
\newcommand*{\restorecolumns}[1][default]%
  {\g@addto@macro\SaveRestoreHook{\restorecolumns[#1]}}
\newcommand*{\aligncolumn}[2]%
  {\g@addto@macro\ColumnHook{\column{#1}{#2}}}

\resethooks

\newcommand{\onelinecommentchars}{\quad-{}- }
\newcommand{\commentbeginchars}{\enskip\{-}
\newcommand{\commentendchars}{-\}\enskip}

\newcommand{\visiblecomments}{%
  \let\onelinecomment=\onelinecommentchars
  \let\commentbegin=\commentbeginchars
  \let\commentend=\commentendchars}

\newcommand{\invisiblecomments}{%
  \let\onelinecomment=\empty
  \let\commentbegin=\empty
  \let\commentend=\empty}

\visiblecomments

\newlength{\blanklineskip}
\setlength{\blanklineskip}{0.66084ex}

\newcommand{\hsindent}[1]{\quad}% default is fixed indentation
\let\hspre\empty
\let\hspost\empty
\newcommand{\NB}{\textbf{NB}}
\newcommand{\Todo}[1]{$\langle$\textbf{To do:}~#1$\rangle$}

\EndFmtInput
\makeatother
%
%
%
%
%
%
% This package provides two environments suitable to take the place
% of hscode, called "plainhscode" and "arrayhscode". 
%
% The plain environment surrounds each code block by vertical space,
% and it uses \abovedisplayskip and \belowdisplayskip to get spacing
% similar to formulas. Note that if these dimensions are changed,
% the spacing around displayed math formulas changes as well.
% All code is indented using \leftskip.
%
% Changed 19.08.2004 to reflect changes in colorcode. Should work with
% CodeGroup.sty.
%
\ReadOnlyOnce{polycode.fmt}%
\makeatletter

\newcommand{\hsnewpar}[1]%
  {{\parskip=0pt\parindent=0pt\par\vskip #1\noindent}}

% can be used, for instance, to redefine the code size, by setting the
% command to \small or something alike
\newcommand{\hscodestyle}{}

% The command \sethscode can be used to switch the code formatting
% behaviour by mapping the hscode environment in the subst directive
% to a new LaTeX environment.

\newcommand{\sethscode}[1]%
  {\expandafter\let\expandafter\hscode\csname #1\endcsname
   \expandafter\let\expandafter\endhscode\csname end#1\endcsname}

% "compatibility" mode restores the non-polycode.fmt layout.

\newenvironment{compathscode}%
  {\par\noindent
   \advance\leftskip\mathindent
   \hscodestyle
   \let\\=\@normalcr
   \let\hspre\(\let\hspost\)%
   \pboxed}%
  {\endpboxed\)%
   \par\noindent
   \ignorespacesafterend}

\newcommand{\compaths}{\sethscode{compathscode}}

% "plain" mode is the proposed default.
% It should now work with \centering.
% This required some changes. The old version
% is still available for reference as oldplainhscode.

\newenvironment{plainhscode}%
  {\hsnewpar\abovedisplayskip
   \advance\leftskip\mathindent
   \hscodestyle
   \let\hspre\(\let\hspost\)%
   \pboxed}%
  {\endpboxed%
   \hsnewpar\belowdisplayskip
   \ignorespacesafterend}

\newenvironment{oldplainhscode}%
  {\hsnewpar\abovedisplayskip
   \advance\leftskip\mathindent
   \hscodestyle
   \let\\=\@normalcr
   \(\pboxed}%
  {\endpboxed\)%
   \hsnewpar\belowdisplayskip
   \ignorespacesafterend}

% Here, we make plainhscode the default environment.

\newcommand{\plainhs}{\sethscode{plainhscode}}
\newcommand{\oldplainhs}{\sethscode{oldplainhscode}}
\plainhs

% The arrayhscode is like plain, but makes use of polytable's
% parray environment which disallows page breaks in code blocks.

\newenvironment{arrayhscode}%
  {\hsnewpar\abovedisplayskip
   \advance\leftskip\mathindent
   \hscodestyle
   \let\\=\@normalcr
   \(\parray}%
  {\endparray\)%
   \hsnewpar\belowdisplayskip
   \ignorespacesafterend}

\newcommand{\arrayhs}{\sethscode{arrayhscode}}

% The mathhscode environment also makes use of polytable's parray 
% environment. It is supposed to be used only inside math mode 
% (I used it to typeset the type rules in my thesis).

\newenvironment{mathhscode}%
  {\parray}{\endparray}

\newcommand{\mathhs}{\sethscode{mathhscode}}

% texths is similar to mathhs, but works in text mode.

\newenvironment{texthscode}%
  {\(\parray}{\endparray\)}

\newcommand{\texths}{\sethscode{texthscode}}

% The framed environment places code in a framed box.

\def\codeframewidth{\arrayrulewidth}
\RequirePackage{calc}

\newenvironment{framedhscode}%
  {\parskip=\abovedisplayskip\par\noindent
   \hscodestyle
   \arrayrulewidth=\codeframewidth
   \tabular{@{}|p{\linewidth-2\arraycolsep-2\arrayrulewidth-2pt}|@{}}%
   \hline\framedhslinecorrect\\{-1.5ex}%
   \let\endoflinesave=\\
   \let\\=\@normalcr
   \(\pboxed}%
  {\endpboxed\)%
   \framedhslinecorrect\endoflinesave{.5ex}\hline
   \endtabular
   \parskip=\belowdisplayskip\par\noindent
   \ignorespacesafterend}

\newcommand{\framedhslinecorrect}[2]%
  {#1[#2]}

\newcommand{\framedhs}{\sethscode{framedhscode}}

% The inlinehscode environment is an experimental environment
% that can be used to typeset displayed code inline.

\newenvironment{inlinehscode}%
  {\(\def\column##1##2{}%
   \let\>\undefined\let\<\undefined\let\\\undefined
   \newcommand\>[1][]{}\newcommand\<[1][]{}\newcommand\\[1][]{}%
   \def\fromto##1##2##3{##3}%
   \def\nextline{}}{\) }%

\newcommand{\inlinehs}{\sethscode{inlinehscode}}

% The joincode environment is a separate environment that
% can be used to surround and thereby connect multiple code
% blocks.

\newenvironment{joincode}%
  {\let\orighscode=\hscode
   \let\origendhscode=\endhscode
   \def\endhscode{\def\hscode{\endgroup\def\@currenvir{hscode}\\}\begingroup}
   %\let\SaveRestoreHook=\empty
   %\let\ColumnHook=\empty
   %\let\resethooks=\empty
   \orighscode\def\hscode{\endgroup\def\@currenvir{hscode}}}%
  {\origendhscode
   \global\let\hscode=\orighscode
   \global\let\endhscode=\origendhscode}%

\makeatother
\EndFmtInput
%

\usepackage{wasysym}
\usepackage[activate={true,nocompatibility},
            final,
            tracking=true,
            kerning=true,
            spacing=true,
            factor=1100,
            stretch=10,
            shrink=10]{microtype}
\microtypecontext{spacing=nonfrench}

\title{Selecting Vulkan Queue Families in Haskell}
\author{idgd}
\date{}

\begin{document}

\maketitle

\section{Problem Statement}

When working with Vulkan, there's lots of small problems to solve in order to render a triangle.
One such example of this is deciding which queue families to use.
Queue families, for the purposes of this article, can be thought of as places to submit work to the GPU.
Each one of these families has a set of capabilities: graphics (rendering images), presentation (showing images on screen), and/or compute (general processing).
The Vulkan specification only guarantees that graphics and presentation will be present; the presence of a compute queue family is optional. Vulkan's API to distinguish those queue families presents us with an array of queue family property objects.
You can query these objects for their capabilities, mostly using some bit manipulation.
Exactly how isn't important, just the end result: the indices into the original array for graphics, present, and compute queue families in a tuple.

With these, we need to create a logical (aka virtual) device.
The function to create that device takes a vector of records, each with an index for which queue family we want this logical device to use.
Our goal is to pass the smallest vector of records into that function as possible, while still covering (at minimum) graphics and presentation capabilities.

Due to the nature of the problem, we don't have one single solution; some solutions are also suboptimal.

\begin{hscode}\SaveRestoreHook
\column{B}{@{}>{\hspre}l<{\hspost}@{}}%
\column{5}{@{}>{\hspre}l<{\hspost}@{}}%
\column{E}{@{}>{\hspre}l<{\hspost}@{}}%
\>[B]{}([\mskip1.5mu \mathrm{1},\mathrm{2}\mskip1.5mu],[\mskip1.5mu \mathrm{2},\mathrm{3}\mskip1.5mu],[\mskip1.5mu \mathrm{3},\mathrm{4}\mskip1.5mu])\to {}\<[E]%
\\
\>[B]{}\hsindent{5}{}\<[5]%
\>[5]{}[\mskip1.5mu \mathrm{1},\mathrm{2},\mathrm{3}\mskip1.5mu]\mbox{\onelinecomment  suboptimal}{}\<[E]%
\\
\>[B]{}\hsindent{5}{}\<[5]%
\>[5]{}[\mskip1.5mu \mathrm{1},\mathrm{3},\mathrm{4}\mskip1.5mu]\mbox{\onelinecomment  suboptimal}{}\<[E]%
\\
\>[B]{}\hsindent{5}{}\<[5]%
\>[5]{}[\mskip1.5mu \mathrm{2},\mathrm{3}\mskip1.5mu]\mbox{\onelinecomment  optimal}{}\<[E]%
\\
\>[B]{}([\mskip1.5mu \mathrm{1}\mskip1.5mu],[\mskip1.5mu \mathrm{2},\mathrm{3}\mskip1.5mu],[\mskip1.5mu \mskip1.5mu])\to {}\<[E]%
\\
\>[B]{}\hsindent{5}{}\<[5]%
\>[5]{}[\mskip1.5mu \mathrm{1},\mathrm{2}\mskip1.5mu]\mbox{\onelinecomment  optimal}{}\<[E]%
\\
\>[B]{}\hsindent{5}{}\<[5]%
\>[5]{}[\mskip1.5mu \mathrm{1},\mathrm{3}\mskip1.5mu]\mbox{\onelinecomment  optimal}{}\<[E]%
\ColumnHook
\end{hscode}\resethooks

Because of this, we're going to pick an arbitrary answer from the best options: minimizing the sum of each vector.
One way to think of the solutions in general is a set of lists; we'll call these solution space (set) and solution (lists), respectively.
Each solution is a list where each element from the three source lists is paired with one element in each other list without duplicates.
Over many possible pairs, we have solution space.

\section{Solution}

Let's start with exploring our module and imports.
We export three functions, which we'll explore next in turn.
We use a state monad later for in-place vector sorting, along with the vector and insertion sort libraries.
We picked insertion sort because our inputs are usually very small; a length of only one or two is not unusual.
We import word to match the types from Vulkan.
Finally, we hide some of the prelude to avoid any shadowing to vector functions.

\begin{hscode}\SaveRestoreHook
\column{B}{@{}>{\hspre}l<{\hspost}@{}}%
\column{5}{@{}>{\hspre}l<{\hspost}@{}}%
\column{E}{@{}>{\hspre}l<{\hspost}@{}}%
\>[B]{}\mathbf{module}\;\Conid{QueueFamilySelect}{}\<[E]%
\\
\>[B]{}\hsindent{5}{}\<[5]%
\>[5]{}(\Varid{selectQueueFamilies}{}\<[E]%
\\
\>[B]{}\hsindent{5}{}\<[5]%
\>[5]{},\Varid{pairgp}{}\<[E]%
\\
\>[B]{}\hsindent{5}{}\<[5]%
\>[5]{},\Varid{pairc}{}\<[E]%
\\
\>[B]{}\hsindent{5}{}\<[5]%
\>[5]{})\;\mathbf{where}{}\<[E]%
\\[\blanklineskip]%
\>[B]{}\mathbf{import}\;\Conid{\Conid{Control}.\Conid{Monad}.ST}{}\<[E]%
\\
\>[B]{}\mathbf{import}\;\Conid{\Conid{Data}.Vector}{}\<[E]%
\\
\>[B]{}\mathbf{import}\;\Conid{\Conid{Data}.\Conid{Vector}.\Conid{Algorithms}.Insertion}{}\<[E]%
\\
\>[B]{}\mathbf{import}\;\Conid{\Conid{Data}.Word}{}\<[E]%
\\
\>[B]{}\mathbf{import}\;\Conid{Prelude}\;\Varid{hiding}\;((\plus ),\Varid{minimum},\Varid{concatMap},\Varid{null},\Varid{filter},\Varid{elem}){}\<[E]%
\ColumnHook
\end{hscode}\resethooks

Given our input tuple, we produce the optimal solution with two function calls.
We'll explore those two next.
The way this function is used, we pair \emph{graphics} and \emph{present}.
We use the result of that to pair with \emph{compute}; at that point, we've found the optimal solution for all three items.

\begin{hscode}\SaveRestoreHook
\column{B}{@{}>{\hspre}l<{\hspost}@{}}%
\column{5}{@{}>{\hspre}l<{\hspost}@{}}%
\column{E}{@{}>{\hspre}l<{\hspost}@{}}%
\>[B]{}\Varid{selectQueueFamilies}\mathbin{::}(\Conid{Vector}\;\Conid{Word32},\Conid{Vector}\;\Conid{Word32},\Conid{Vector}\;\Conid{Word32}){}\<[E]%
\\
\>[B]{}\hsindent{5}{}\<[5]%
\>[5]{}\to \Conid{Vector}\;\Conid{Word32}{}\<[E]%
\\
\>[B]{}\Varid{selectQueueFamilies}\;(\Varid{graphics},\Varid{present},\Varid{compute})\mathrel{=}{}\<[E]%
\\
\>[B]{}\hsindent{5}{}\<[5]%
\>[5]{}\Varid{pairc}\;(\Varid{pairgp}\;\Varid{graphics}\;\Varid{present})\;\Varid{compute}{}\<[E]%
\ColumnHook
\end{hscode}\resethooks

We'll look at \emph{pairgp} first.
We take a two vectors, \emph{a} and \emph{b}.
This pair is constrained in that it is guaranteed never to be empty, so we can safely define this function partially.
These vectors are passed into \emph{concatMap}, along with a lambda.
\emph{concatMap} will map the given function over the vector, which will generate another vector, and then concatenate the results.
The outer lambda maps another lambda over \emph{b}; the inner lambda takes each element of \emph{a} and appends it to a vector with each element of \emph{b}.
We end up with every possible pairing of each element in both vectors.

We take the minimum from this list, and pass it into a custom function which will sort it and remove any duplicates.
This prevents us from presenting any incorrect answers like \emph{([0], [0]) $\to$ [0,0]}.
Once we have this, we can move on to pairing this result and the (possibly empty) compute vector.

\begin{hscode}\SaveRestoreHook
\column{B}{@{}>{\hspre}l<{\hspost}@{}}%
\column{5}{@{}>{\hspre}l<{\hspost}@{}}%
\column{E}{@{}>{\hspre}l<{\hspost}@{}}%
\>[B]{}\Varid{pairgp}\mathbin{::}\Conid{Vector}\;\Conid{Word32}\to \Conid{Vector}\;\Conid{Word32}\to \Conid{Vector}\;\Conid{Word32}{}\<[E]%
\\
\>[B]{}\Varid{pairgp}\;\Varid{a}\;\Varid{b}\mathrel{=}\Varid{vUniqSort}\mathbin{\$}\Varid{minimum}\mathbin{\$}{}\<[E]%
\\
\>[B]{}\hsindent{5}{}\<[5]%
\>[5]{}\Varid{concatMap}\;(\lambda \Varid{i}\to \Varid{fmap}\;(\lambda \Varid{j}\to \Varid{fromList}\;[\mskip1.5mu \Varid{i},\Varid{j}\mskip1.5mu])\;\Varid{b})\;\Varid{a}{}\<[E]%
\ColumnHook
\end{hscode}\resethooks

Here, only \emph{b}, as the vector for compute queue family indices, could be null.
We check for that, and shortcut to just \emph{a} if that's the case.
Otherwise, we check if any elements from \emph{a} exist in \emph{b}.
If not, that means we need to gather something from \emph{b}, so we pick the minimum.
Finally, if we don't need to, we return \emph{a} by itself, since it sufficiently covers our input.

\begin{hscode}\SaveRestoreHook
\column{B}{@{}>{\hspre}l<{\hspost}@{}}%
\column{5}{@{}>{\hspre}l<{\hspost}@{}}%
\column{E}{@{}>{\hspre}l<{\hspost}@{}}%
\>[B]{}\Varid{pairc}\mathbin{::}\Conid{Vector}\;\Conid{Word32}\to \Conid{Vector}\;\Conid{Word32}\to \Conid{Vector}\;\Conid{Word32}{}\<[E]%
\\
\>[B]{}\Varid{pairc}\;\Varid{a}\;\Varid{b}\mathrel{=}\mathbf{if}\;\Varid{null}\;\Varid{b}{}\<[E]%
\\
\>[B]{}\hsindent{5}{}\<[5]%
\>[5]{}\mathbf{then}\;\Varid{a}{}\<[E]%
\\
\>[B]{}\hsindent{5}{}\<[5]%
\>[5]{}\mathbf{else}\;\mathbf{if}\;\Varid{null}\mathbin{\$}\Varid{filter}\;(\lambda \Varid{i}\to \Varid{elem}\;\Varid{i}\;\Varid{b})\;\Varid{a}{}\<[E]%
\\
\>[B]{}\hsindent{5}{}\<[5]%
\>[5]{}\mathbf{then}\;\Varid{cons}\;(\Varid{minimum}\;\Varid{b})\;\Varid{a}{}\<[E]%
\\
\>[B]{}\hsindent{5}{}\<[5]%
\>[5]{}\mathbf{else}\;\Varid{a}{}\<[E]%
\ColumnHook
\end{hscode}\resethooks

Let's finally take a look at our combination nub (remove duplicates) and sort.
We use the \emph{vector-algorithms} package to sort our list.
Since this package does in-place sorting, we use the strict state thread monad to `thaw' (make mutable), sort, and `freeze' (make immutable) our newly mutable vector.
Finally, we return the output of \emph{uniq}, which removes consecutive duplicates.
Since our vector is sorted, we can guarantee that \emph{uniq} in this case is equivalent to a nub.

\begin{hscode}\SaveRestoreHook
\column{B}{@{}>{\hspre}l<{\hspost}@{}}%
\column{5}{@{}>{\hspre}l<{\hspost}@{}}%
\column{E}{@{}>{\hspre}l<{\hspost}@{}}%
\>[B]{}\Varid{vUniqSort}\mathbin{::}\Conid{Vector}\;\Conid{Word32}\to \Conid{Vector}\;\Conid{Word32}{}\<[E]%
\\
\>[B]{}\Varid{vUniqSort}\;\Varid{a}\mathrel{=}\Varid{runST}\mathbin{\$}\mathbf{do}{}\<[E]%
\\
\>[B]{}\hsindent{5}{}\<[5]%
\>[5]{}\Varid{t}\leftarrow \Varid{thaw}\;\Varid{a}{}\<[E]%
\\
\>[B]{}\hsindent{5}{}\<[5]%
\>[5]{}\Varid{sort}\;\Varid{t}{}\<[E]%
\\
\>[B]{}\hsindent{5}{}\<[5]%
\>[5]{}\Varid{f}\leftarrow \Varid{freeze}\;\Varid{t}{}\<[E]%
\\
\>[B]{}\hsindent{5}{}\<[5]%
\>[5]{}\Varid{return}\mathbin{\$}\Varid{uniq}\;\Varid{f}{}\<[E]%
\ColumnHook
\end{hscode}\resethooks

\section{Testing}

To test this, we need to define our inputs and outputs.
Our inputs are:

\begin{itemize}
\item Three word32 vectors
\item The first two of the three lists are guaranteed to have one item in them
\item The last one is not guaranteed to have any
\end{itemize}

Our outputs are:

\begin{itemize}
\item One word32 vector
\item The first two input lists have at least one element represented
\item At most one element of the third input list is represented
\item No duplicates
\item At most three items
\end{itemize}

We obviously have our library; otherwise we couldn't test it.
Also here are one function from the list type, along with vector and word.
We also hide some functions from prelude in order to simplify vector calls.
Finally, we import QuickCheck in order to automate our testing.

\begin{hscode}\SaveRestoreHook
\column{B}{@{}>{\hspre}l<{\hspost}@{}}%
\column{E}{@{}>{\hspre}l<{\hspost}@{}}%
\>[B]{}\mathbf{import}\;\Conid{\Conid{Data}.List}\;(\Varid{nub}){}\<[E]%
\\
\>[B]{}\mathbf{import}\;\Conid{\Conid{Data}.Vector}{}\<[E]%
\\
\>[B]{}\mathbf{import}\;\Conid{\Conid{Data}.Word}{}\<[E]%
\\
\>[B]{}\mathbf{import}\;\Conid{Prelude}\;\Varid{hiding}\;(\Varid{elem},\Varid{any},\Varid{null},\Varid{length},\Varid{minimum}){}\<[E]%
\\
\>[B]{}\mathbf{import}\;\Conid{QueueFamilySelect}{}\<[E]%
\\
\>[B]{}\mathbf{import}\;\Conid{\Conid{Test}.QuickCheck}{}\<[E]%
\\
\>[B]{}\mathbf{import}\;\Conid{\Conid{Test}.\Conid{QuickCheck}.\Conid{Instances}.Vector}\;(){}\<[E]%
\ColumnHook
\end{hscode}\resethooks

Our first test will be on the innermost function in our library, \emph{pairgp}.
We first constrain the input so we throw away any null inputs; we're guaranteed that graphics and presentation queue families will have an entry, so we reflect that here.
Next, we use our \emph{chk} function to validate that each input list is represented.
We'll explore that next.

\begin{hscode}\SaveRestoreHook
\column{B}{@{}>{\hspre}l<{\hspost}@{}}%
\column{5}{@{}>{\hspre}l<{\hspost}@{}}%
\column{9}{@{}>{\hspre}l<{\hspost}@{}}%
\column{E}{@{}>{\hspre}l<{\hspost}@{}}%
\>[B]{}\Varid{pairgpElements}\mathbin{::}(\Conid{Vector}\;\Conid{Word32},\Conid{Vector}\;\Conid{Word32})\to \Conid{Property}{}\<[E]%
\\
\>[B]{}\Varid{pairgpElements}\;(\Varid{a},\Varid{b})\mathrel{=}{}\<[E]%
\\
\>[B]{}\hsindent{5}{}\<[5]%
\>[5]{}\mathbf{if}\;(\Varid{null}\;\Varid{a}\mathrel{\vee}\Varid{null}\;\Varid{b}){}\<[E]%
\\
\>[B]{}\hsindent{5}{}\<[5]%
\>[5]{}\mathbf{then}\;\Varid{property}\;\Conid{Discard}{}\<[E]%
\\
\>[B]{}\hsindent{5}{}\<[5]%
\>[5]{}\mathbf{else}\;\Varid{property}\;(\Varid{\Conid{Prelude}.all}\;\Varid{id}\;[\mskip1.5mu \Varid{chka},\Varid{chkb}\mskip1.5mu]){}\<[E]%
\\
\>[B]{}\hsindent{5}{}\<[5]%
\>[5]{}\mathbf{where}{}\<[E]%
\\
\>[5]{}\hsindent{4}{}\<[9]%
\>[9]{}\Varid{d}\mathbin{::}\Conid{Vector}\;\Conid{Word32}{}\<[E]%
\\
\>[5]{}\hsindent{4}{}\<[9]%
\>[9]{}\Varid{d}\mathrel{=}\Varid{pairgp}\;\Varid{a}\;\Varid{b}{}\<[E]%
\\
\>[5]{}\hsindent{4}{}\<[9]%
\>[9]{}\Varid{chka}\mathbin{::}\Conid{Bool}{}\<[E]%
\\
\>[5]{}\hsindent{4}{}\<[9]%
\>[9]{}\Varid{chka}\mathrel{=}\Varid{chk}\;\Varid{d}\;\Varid{a}{}\<[E]%
\\
\>[5]{}\hsindent{4}{}\<[9]%
\>[9]{}\Varid{chkb}\mathbin{::}\Conid{Bool}{}\<[E]%
\\
\>[5]{}\hsindent{4}{}\<[9]%
\>[9]{}\Varid{chkb}\mathrel{=}\Varid{chk}\;\Varid{d}\;\Varid{b}{}\<[E]%
\ColumnHook
\end{hscode}\resethooks

\emph{chk} is a little shared utility function that checks to see if any element from \emph{x} exists in \emph{y}.
Above, and further on, we use it to validate that the output of our functions still has elements from our input.

\begin{hscode}\SaveRestoreHook
\column{B}{@{}>{\hspre}l<{\hspost}@{}}%
\column{E}{@{}>{\hspre}l<{\hspost}@{}}%
\>[B]{}\Varid{chk}\mathbin{::}\Conid{Vector}\;\Conid{Word32}\to \Conid{Vector}\;\Conid{Word32}\to \Conid{Bool}{}\<[E]%
\\
\>[B]{}\Varid{chk}\;\Varid{x}\;\Varid{y}\mathrel{=}\Varid{any}\;\Varid{id}\mathbin{\$}\Varid{fmap}\;(\lambda \Varid{i}\to \Varid{elem}\;\Varid{i}\;\Varid{y})\;\Varid{x}{}\<[E]%
\ColumnHook
\end{hscode}\resethooks

We'll do the same again with \emph{pairc}; the second input can be null here, so we change our constraints to match.
We'll chain our if statements a bit further here as well, in order to check all possible null inputs.
A null \emph{b} should guarantee \emph{a} as the output, so we check that; otherwise, this function should operate exactly the same as \emph{pairgp}.

\begin{hscode}\SaveRestoreHook
\column{B}{@{}>{\hspre}l<{\hspost}@{}}%
\column{5}{@{}>{\hspre}l<{\hspost}@{}}%
\column{9}{@{}>{\hspre}l<{\hspost}@{}}%
\column{E}{@{}>{\hspre}l<{\hspost}@{}}%
\>[B]{}\Varid{paircElements}\mathbin{::}(\Conid{Vector}\;\Conid{Word32},\Conid{Vector}\;\Conid{Word32})\to \Conid{Property}{}\<[E]%
\\
\>[B]{}\Varid{paircElements}\;(\Varid{a},\Varid{b})\mathrel{=}{}\<[E]%
\\
\>[B]{}\hsindent{5}{}\<[5]%
\>[5]{}\mathbf{if}\;\Varid{null}\;\Varid{a}{}\<[E]%
\\
\>[B]{}\hsindent{5}{}\<[5]%
\>[5]{}\mathbf{then}\;\Varid{property}\;\Conid{Discard}{}\<[E]%
\\
\>[B]{}\hsindent{5}{}\<[5]%
\>[5]{}\mathbf{else}\;\mathbf{if}\;\Varid{null}\;\Varid{b}{}\<[E]%
\\
\>[B]{}\hsindent{5}{}\<[5]%
\>[5]{}\mathbf{then}\;\Varid{pairc}\;\Varid{a}\;\Varid{b}\mathbin{===}\Varid{a}{}\<[E]%
\\
\>[B]{}\hsindent{5}{}\<[5]%
\>[5]{}\mathbf{else}\;\Varid{property}\;(\Varid{\Conid{Prelude}.all}\;\Varid{id}\;[\mskip1.5mu \Varid{chka},\Varid{chkb}\mskip1.5mu]){}\<[E]%
\\
\>[B]{}\hsindent{5}{}\<[5]%
\>[5]{}\mathbf{where}{}\<[E]%
\\
\>[5]{}\hsindent{4}{}\<[9]%
\>[9]{}\Varid{d}\mathbin{::}\Conid{Vector}\;\Conid{Word32}{}\<[E]%
\\
\>[5]{}\hsindent{4}{}\<[9]%
\>[9]{}\Varid{d}\mathrel{=}\Varid{pairc}\;\Varid{a}\;\Varid{b}{}\<[E]%
\\
\>[5]{}\hsindent{4}{}\<[9]%
\>[9]{}\Varid{chka}\mathbin{::}\Conid{Bool}{}\<[E]%
\\
\>[5]{}\hsindent{4}{}\<[9]%
\>[9]{}\Varid{chka}\mathrel{=}\Varid{chk}\;\Varid{d}\;\Varid{a}{}\<[E]%
\\
\>[5]{}\hsindent{4}{}\<[9]%
\>[9]{}\Varid{chkb}\mathbin{::}\Conid{Bool}{}\<[E]%
\\
\>[5]{}\hsindent{4}{}\<[9]%
\>[9]{}\Varid{chkb}\mathrel{=}\Varid{chk}\;\Varid{d}\;\Varid{b}{}\<[E]%
\ColumnHook
\end{hscode}\resethooks

This same check is done here, but we permit the third input to be null, as it could be in real usage.

\begin{hscode}\SaveRestoreHook
\column{B}{@{}>{\hspre}l<{\hspost}@{}}%
\column{5}{@{}>{\hspre}l<{\hspost}@{}}%
\column{9}{@{}>{\hspre}l<{\hspost}@{}}%
\column{12}{@{}>{\hspre}l<{\hspost}@{}}%
\column{E}{@{}>{\hspre}l<{\hspost}@{}}%
\>[B]{}\Varid{selectQueueFamiliesElements}\mathbin{::}{}\<[E]%
\\
\>[B]{}\hsindent{5}{}\<[5]%
\>[5]{}(\Conid{Vector}\;\Conid{Word32},\Conid{Vector}\;\Conid{Word32},\Conid{Vector}\;\Conid{Word32}){}\<[E]%
\\
\>[B]{}\hsindent{5}{}\<[5]%
\>[5]{}\to \Conid{Property}{}\<[E]%
\\
\>[B]{}\Varid{selectQueueFamiliesElements}\;(\Varid{a},\Varid{b},\Varid{c})\mathrel{=}{}\<[E]%
\\
\>[B]{}\hsindent{5}{}\<[5]%
\>[5]{}\mathbf{if}\;(\Varid{null}\;\Varid{a}\mathrel{\vee}\Varid{null}\;\Varid{b}){}\<[E]%
\\
\>[B]{}\hsindent{5}{}\<[5]%
\>[5]{}\mathbf{then}\;\Varid{property}\;\Conid{Discard}{}\<[E]%
\\
\>[B]{}\hsindent{5}{}\<[5]%
\>[5]{}\mathbf{else}\;\Varid{property}\mathbin{\$}(\Varid{\Conid{Prelude}.all}\;\Varid{id}\;[\mskip1.5mu \Varid{chka},\Varid{chkb}\mskip1.5mu]){}\<[E]%
\\
\>[5]{}\hsindent{4}{}\<[9]%
\>[9]{}\mathrel{\wedge}\mathbf{if}\;\Varid{null}\;\Varid{c}{}\<[E]%
\\
\>[9]{}\hsindent{3}{}\<[12]%
\>[12]{}\mathbf{then}\;\Conid{True}{}\<[E]%
\\
\>[9]{}\hsindent{3}{}\<[12]%
\>[12]{}\mathbf{else}\;\Varid{chkc}{}\<[E]%
\\
\>[B]{}\hsindent{5}{}\<[5]%
\>[5]{}\mathbf{where}{}\<[E]%
\\
\>[5]{}\hsindent{4}{}\<[9]%
\>[9]{}\Varid{d}\mathbin{::}\Conid{Vector}\;\Conid{Word32}{}\<[E]%
\\
\>[5]{}\hsindent{4}{}\<[9]%
\>[9]{}\Varid{d}\mathrel{=}\Varid{selectQueueFamilies}\;(\Varid{a},\Varid{b},\Varid{c}){}\<[E]%
\\
\>[5]{}\hsindent{4}{}\<[9]%
\>[9]{}\Varid{chka}\mathbin{::}\Conid{Bool}{}\<[E]%
\\
\>[5]{}\hsindent{4}{}\<[9]%
\>[9]{}\Varid{chka}\mathrel{=}\Varid{chk}\;\Varid{d}\;\Varid{a}{}\<[E]%
\\
\>[5]{}\hsindent{4}{}\<[9]%
\>[9]{}\Varid{chkb}\mathbin{::}\Conid{Bool}{}\<[E]%
\\
\>[5]{}\hsindent{4}{}\<[9]%
\>[9]{}\Varid{chkb}\mathrel{=}\Varid{chk}\;\Varid{d}\;\Varid{b}{}\<[E]%
\\
\>[5]{}\hsindent{4}{}\<[9]%
\>[9]{}\Varid{chkc}\mathbin{::}\Conid{Bool}{}\<[E]%
\\
\>[5]{}\hsindent{4}{}\<[9]%
\>[9]{}\Varid{chkc}\mathrel{=}\Varid{chk}\;\Varid{d}\;\Varid{c}{}\<[E]%
\ColumnHook
\end{hscode}\resethooks

We also want to know if the length of the output is less than three.
Because our end goal is to select one from each, and deduplicate, we should always be less than three at the end.

\begin{hscode}\SaveRestoreHook
\column{B}{@{}>{\hspre}l<{\hspost}@{}}%
\column{5}{@{}>{\hspre}l<{\hspost}@{}}%
\column{E}{@{}>{\hspre}l<{\hspost}@{}}%
\>[B]{}\Varid{selectQueueFamiliesLength}\mathbin{::}{}\<[E]%
\\
\>[B]{}\hsindent{5}{}\<[5]%
\>[5]{}(\Conid{Vector}\;\Conid{Word32},\Conid{Vector}\;\Conid{Word32},\Conid{Vector}\;\Conid{Word32}){}\<[E]%
\\
\>[B]{}\hsindent{5}{}\<[5]%
\>[5]{}\to \Conid{Property}{}\<[E]%
\\
\>[B]{}\Varid{selectQueueFamiliesLength}\;(\Varid{a},\Varid{b},\Varid{c})\mathrel{=}{}\<[E]%
\\
\>[B]{}\hsindent{5}{}\<[5]%
\>[5]{}\mathbf{if}\;(\Varid{null}\;\Varid{a}\mathrel{\vee}\Varid{null}\;\Varid{b}){}\<[E]%
\\
\>[B]{}\hsindent{5}{}\<[5]%
\>[5]{}\mathbf{then}\;\Varid{property}\;\Conid{Discard}{}\<[E]%
\\
\>[B]{}\hsindent{5}{}\<[5]%
\>[5]{}\mathbf{else}\;\Varid{property}\mathbin{\$}\Varid{length}\;(\Varid{selectQueueFamilies}\;(\Varid{a},\Varid{b},\Varid{c}))\leq \mathrm{3}{}\<[E]%
\ColumnHook
\end{hscode}\resethooks

Similarly, there shouldn't be any duplicates in the output, and it should be sorted.
A nub of our output should be equivalent to the output.

\begin{hscode}\SaveRestoreHook
\column{B}{@{}>{\hspre}l<{\hspost}@{}}%
\column{5}{@{}>{\hspre}l<{\hspost}@{}}%
\column{9}{@{}>{\hspre}l<{\hspost}@{}}%
\column{E}{@{}>{\hspre}l<{\hspost}@{}}%
\>[B]{}\Varid{selectQueueFamiliesDupes}\mathbin{::}{}\<[E]%
\\
\>[B]{}\hsindent{5}{}\<[5]%
\>[5]{}(\Conid{Vector}\;\Conid{Word32},\Conid{Vector}\;\Conid{Word32},\Conid{Vector}\;\Conid{Word32}){}\<[E]%
\\
\>[B]{}\hsindent{5}{}\<[5]%
\>[5]{}\to \Conid{Property}{}\<[E]%
\\
\>[B]{}\Varid{selectQueueFamiliesDupes}\;(\Varid{a},\Varid{b},\Varid{c})\mathrel{=}{}\<[E]%
\\
\>[B]{}\hsindent{5}{}\<[5]%
\>[5]{}\mathbf{if}\;(\Varid{null}\;\Varid{a}\mathrel{\vee}\Varid{null}\;\Varid{b}){}\<[E]%
\\
\>[B]{}\hsindent{5}{}\<[5]%
\>[5]{}\mathbf{then}\;\Varid{property}\;\Conid{Discard}{}\<[E]%
\\
\>[B]{}\hsindent{5}{}\<[5]%
\>[5]{}\mathbf{else}\;(\Varid{fromList}\mathbin{\circ}\Varid{nub}\mathbin{\circ}\Varid{toList}\mathbin{\$}\Varid{selectQueueFamilies}\;(\Varid{a},\Varid{b},\Varid{c})){}\<[E]%
\\
\>[5]{}\hsindent{4}{}\<[9]%
\>[9]{}\mathbin{===}(\Varid{selectQueueFamilies}\;(\Varid{a},\Varid{b},\Varid{c})){}\<[E]%
\ColumnHook
\end{hscode}\resethooks

Finally, let's run the complete set of tests.
We've tested both inner functions, and the three properties of our output that we know are invariant.

\begin{hscode}\SaveRestoreHook
\column{B}{@{}>{\hspre}l<{\hspost}@{}}%
\column{5}{@{}>{\hspre}l<{\hspost}@{}}%
\column{E}{@{}>{\hspre}l<{\hspost}@{}}%
\>[B]{}\Varid{main}\mathbin{::}\Conid{IO}\;(){}\<[E]%
\\
\>[B]{}\Varid{main}\mathrel{=}\mathbf{do}{}\<[E]%
\\
\>[B]{}\hsindent{5}{}\<[5]%
\>[5]{}\Varid{quickCheck}\;\Varid{pairgpElements}{}\<[E]%
\\
\>[B]{}\hsindent{5}{}\<[5]%
\>[5]{}\Varid{quickCheck}\;\Varid{paircElements}{}\<[E]%
\\
\>[B]{}\hsindent{5}{}\<[5]%
\>[5]{}\Varid{quickCheck}\;\Varid{selectQueueFamiliesElements}{}\<[E]%
\\
\>[B]{}\hsindent{5}{}\<[5]%
\>[5]{}\Varid{quickCheck}\;\Varid{selectQueueFamiliesLength}{}\<[E]%
\\
\>[B]{}\hsindent{5}{}\<[5]%
\>[5]{}\Varid{quickCheck}\;\Varid{selectQueueFamiliesDupes}{}\<[E]%
\\
\>[B]{}\hsindent{5}{}\<[5]%
\>[5]{}\Varid{return}\;(){}\<[E]%
\ColumnHook
\end{hscode}\resethooks

Now that we've done all that, I should note that most real-world inputs to this whole function will be \emph{([0,1],[0],[0])}.
If we were going for practicality, and not overengineered correctness, we could've done something much simpler.
But this was much more fun. \smiley


\section{How was this document generated?}

We use a custom setup in our package.yaml file, which stack uses to execute the below Haskell (in Setup.hs).
It acts equivalently to a shell script; we could use the pandoc package directly, but lhs2TeX provides no such in-language support; so we standardized on shell.

\begin{tabbing}\ttfamily
~import~Distribution\char46{}Simple\\
\ttfamily ~import~System\char46{}Process\\
\ttfamily ~\\
\ttfamily ~main~\char61{}~do\\
\ttfamily ~~~createProcess~\char40{}shell\\
\ttfamily ~~~~~\char34{}lhs2TeX~src\char47{}Lib\char46{}lhs~\char62{}~Lib\char46{}tex~\char38{}\char38{}~texi2pdf~\char45{}c~\char45{}q~Lib\char46{}tex\char34{}\char41{}\\
\ttfamily ~~~createProcess~\char40{}shell\\
\ttfamily ~~~~~\char34{}pandoc~\char45{}\char45{}from\char61{}LaTeX\char43{}lhs~\char45{}\char45{}to\char61{}plain~src\char47{}Lib\char46{}lhs~\char45{}o~Lib\char46{}txt\char34{}\char41{}\\
\ttfamily ~~~createProcess~\char40{}shell\\
\ttfamily ~~~~~\char34{}pandoc~\char45{}\char45{}from\char61{}LaTeX\char43{}lhs~\char45{}\char45{}to\char61{}plain~test\char47{}Spec\char46{}lhs~\char45{}o~Spec\char46{}txt\char34{}\char41{}\\
\ttfamily ~~~defaultMain
\end{tabbing}

I will admit I also used some manual editing to the HTML files.
This text in the top level of our package.yaml file is what guarantees Setup.hs is run.

\begin{tabbing}\ttfamily
~build\char45{}type\char58{}~Custom\\
\ttfamily ~custom\char45{}setup\char58{}\\
\ttfamily ~~~dependencies\char58{}~base~\char62{}\char61{}~4\char46{}7~\char38{}\char38{}~\char60{}~5\char44{}~Cabal\char44{}~process
\end{tabbing}

\end{document}
